%% FIRST RENAME THIS FILE <yoursurname>.tex. 
%% BEFORE COMPLETING THIS TEMPLATE, SEE THE "READ ME" SECTION 
%% BELOW FOR INSTRUCTIONS. 
%% TO PROCESS THIS FILE YOU WILL NEED TO DOWNLOAD asl.cls from 
%% http://aslonline.org/abstractresources.html. 


% \documentclass[bsl,meeting]{asl}
\documentclass[bsl,meeting,bibalpha]{asl}
\AbstractsOn
\pagestyle{plain}
\def\urladdr#1{\endgraf\noindent{\it URL Address}: {\tt #1}.}


\newcommand{\NP}{}
%\usepackage{verbatim}


% header-includes

\begin{document}
\thispagestyle{empty}

% include-before

%% BEGIN INSERTING YOUR ABSTRACT DIRECTLY BELOW; 
%% SEE INSTRUCTIONS (1), (2), (3), and (4) FOR PROPER FORMATS

\NP  
\absauth{Hongwei Xi and Hanwen Wu}
\meettitle{Sample Title}
\affil{Dept. of Comp. Sci., Boston Univ., 111 Cummington Mall, Boston MA, US}
\meetemail{hwxi,hwwu@cs.bu.edu}


%% INSERT TEXT OF ABSTRACT DIRECTLY BELOW

Session types offer a type-based discipline for enforcing communication
protocols in distributed programming. We have previously formalized
simple session types in the setting of multi-threaded
\(\lambda\)-calculus with linear types. In this work, we build upon our
earlier work by presenting a form of dependent session types (of
DML-style). The type system we formulate provides linearity and duality
guarantees with no need for any runtime checks or special encodings. Our
formulation of dependent session types is the first of its kind, and it
is particularly suitable for practical implementation. As an example, we
describe one implementation written in ATS that compiles to an
Erlang/Elixir back-end. \cite{Xi:2017wv}

% \begin{thebibliography}{10}

%% INSERT YOUR BIBLIOGRAPHIC ENTRIES HERE; 
%% SEE (4) BELOW FOR PROPER FORMAT.
%% EACH ENTRY MUST BEGIN WITH \bibitem{citation key}
%%
%% IF THERE ARE NO ENTRIES  
%% DELETE THE LINE ABOVE (\begin{thebibliography}{20}) 
%% AND THE LINE BELOW (\end{thebibliography})

% \end{thebibliography}

\bibliographystyle{asl}
\begin{thebibliography}{9}

\bibfitem{Xi:2017wv}
{\guy{H.}{Hongwei}{}{Xi}{} and \guy{H.}{Hanwen}{}{Wu}{}}
{2017}{}{0}
{}{}
\guysmagic{{\scshape Hongwei Xi \biband{} Hanwen Wu}}  {\em {Multirole Logic
  (Extended Abstract).}}, {\bfseries\itshape CoRR}\yearmagic{,}{(2017)}.
\TheSortKeyIs{xi  hongwei   wu  hanwen    2017    multirole logic extended
  abstract}

\end{thebibliography}


\vspace*{-0.5\baselineskip}
% this space adjustment is usually necessary after a bibliography


% include-after

\end{document}


%% READ ME
%% READ ME
%% READ ME

% INSTRUCTIONS FOR SUPPLYING INFORMATION IN THE CORRECT FORMAT: 

% 1. Author names are listed as First Last, First Last, and First Last.

% \absauth{FirstName1 LastName1, FirstName2 LastName2, and FirstName3 LastName3}


% 2. Titles of abstracts have ONLY the first letter capitalized,
% except for Proper Nouns.

% \meettitle{Title of abstract with initial capital letter only, except for
% Proper Nouns} 


% 3. Affiliations and email addresses for authors of abstracts are
%   listed separately.

% First author's affiliation
% \affil{Department, University, Street Address, Country}
% \meetemail{First author's email}
%%% NOTE: email required for at least one author
% \urladdr{OPTIONAL}
%
% Second author's affiliation
% \affil{Department, University, Street Address, Country}
% \meetemail{Second author's email}
% \urladdr{OPTIONAL}
%
% Third author's affiliation
% \affil{Department, University, Street Address, Country}
% Second author's email
% \meetemail{Third author's email}
% \urladdr{OPTIONAL}


% 4. Bibliographic Entries

% %%%% IF references are submitted with abstract,
% %%%% please use the following formats

% %%% For a Journal article
% \bibitem{cite1}
% {\scshape Author's Name},
% {\itshape Title of article},
% {\bfseries\itshape Journal name spelled out, no abbreviations},
% vol.~XX (XXXX), no.~X, pp.~XXX--XXX.

% %%% For a Journal article by the same authors as above,
% %%% i.e., authors in cite1 are the same for cite2
% \bibitem{cite2}
% \bysame
% {\itshape Title of article},
% {\bfseries\itshape Journal},
% vol.~XX (XXXX), no.~X, pp.~XX--XXX.

% %%% For a book
% \bibitem{cite3}
% {\scshape Author's Name},
% {\bfseries\itshape Title of book},
% Name of series,
% Publisher,
% Year.

% %%% For an article in proceedings
% \bibitem{cite4}
% {\scshape Author's Name},
% {\itshape Title of article},
% {\bfseries\itshape Name of proceedings}
% (Address of meeting),
% (First Last and First2 Last2, editors),
% vol.~X,
% Publisher,
% Year,
% pp.~X--XX.

% %%% For an article in a collection
% \bibitem{cite5}
% {\scshape Author's Name},
% {\itshape Title of article},
% {\bfseries\itshape Book title}
% (First Last and First2 Last2, editors),
% Publisher,
% Publisher's address,
% Year,
% pp.~X--XX.

% %%% An edited book
% \bibitem{cite6}
% Author's name, editor. % No special font used here
% {\bfseries\itshape Title of book},
% Publisher,
% Publisher's address,
% Year.

